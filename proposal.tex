% Template from Usenix
\documentclass[letterpaper,twocolumn,10pt]{article}
\usepackage{usenix,epsfig,endnotes}
\usepackage{graphicx}

\begin{document}

% Don't output date
\date{}

% Title
\title{\Large \textbf{
\includegraphics[height=\baselineskip]{title} \\
Project Proposal } \\ \vspace{0.025 in} \large \normalfont
CSE 227: Computer Security - Spring 2017 \\ \textit{
University of California San Diego
}}

% Authors
\author{
{\rm Chen Lai}\\
\normalfont{\texttt{chl588@ucsd.edu}}
\and
{\rm Zhongrong Jian}\\
\normalfont{\texttt{zhjian@ucsd.edu}}
\and
{\rm Juan Sidrach}\\
\normalfont{\texttt{jsidrach@ucsd.edu}}
}

\maketitle

\section{What}

- Explain the attack
  - When/how were IDN introduced
  - Homograph letters
  - How to use this to the attackers advantage

Domain name were designed to support ASCII character. Internationalized domain name(IDN) was proposed in December 1996 by Martin Dürst for the purpose of letting non-English speaking people use Internet without restriction[1]. The solution is to implement unicode mapping every character in different language to a unique number. Homograph letters, however, could be a potential vulnerability. For example, Cyrillic letter 'a' can look identical to Latin letter 'a'. Attackers can use the website "www.apple.com" where 'a' actually is a Cyrillic letter to attack users. In some other language, like Chinese, there exists many homographs between traditional Chinese and simplified Chinese.

- Research
  - On websites
    - Top 500 Alexa Websites
    - Try all possible variations
    - Classify: 404, redirect (legitimate), malicious, unrelated
    - Evaluation
  - On browsers
    - % Affected, version affected
    - Screenshots before/after CVE (url bar, hover link)
    - Other possible policies: why they may work or they won't
      - Maybe domain providers should not allow IDNs close to real names?
      - Whitelist/blacklist

- Conclusion
  - Impact of the vulnerability based on empirical data and % of users still vulnerable
  - Possible solution for this vulnerability. How does company solve this problem.

\section{Why}
- Relatively new CVE discovered because this (https://bugs.chromium.org/p/chromium/issues/detail?id=683314)
- How something beneficial for the users IDN could be turned against them
- Hard for regular users to distinguish between real/fake sites
- Easy to collect data to evaluate impact

\section{Potential Issues}
- The potential impact, it may be actually really low
- If we will find not legitimate sites by substitution (sample size may be low)
- If there are a lot of illegitime sites, we would need to check manually if they are just a redirect or a scam, maybe we will need to find an automatic way



\section{Resources}
Resources
- We believe for the TOP500 sites + variations of each one we can check it with our own computers/crawlers, same for different browsers


\end{document}
