\documentclass[letterpaper,twocolumn,10pt]{article}
\usepackage{url,epsfig}
\begin{document}

% Don't output date
\date{}

% Title
\title{\Large
\includegraphics[height=\baselineskip]{title}
\\ \vspace{0.65mm} \large \normalfont
CSE 227: Computer Security - Spring 2017 \\ \textit{
University of California San Diego
}}

% Authors
\author{
{\rm Chen Lai}\\
\normalfont{\texttt{chl588@ucsd.edu}}
\and
{\rm Zhongrong Jian}\\
\normalfont{\texttt{zhjian@ucsd.edu}}
\and
{\rm J. Sidrach}\\
\normalfont{\texttt{jsidrach@ucsd.edu}}
}

\maketitle

\abstract{TODO}

\section{Introduction}
TODO
- Introduction to the problem
- Enumeration of sections this paper talks about

\section{Background}
TODO
- DNS: explanation
- IDN: history, explanation
- Browsers: display of IDNs, algorithms

\section{Related Work}
TODO
- Brief analysis of previous papers on the same topic

\section{Methodology}
Our data collection involves two primary sources.
The first one is a snapshot of the \texttt{.com} and \texttt{.net} domain zone, from now on referenced as \textit{.com snapshot}.
This snapshot was provided by Verisign\footnote{\url{https://www.verisign.com/en_US/channel-resources/domain-registry-products/zone-file/index.xhtml}}, and it is dated on 2017/05/01.
It contains, mostly, the name servers of all \texttt{.com} and \texttt{.net} domains.
For the purposes of this project, only the \texttt{.com} domains were considered.

The second data source is the \textit{Alexa Top 1 million sites ranking}\footnote{\url{https://www.alexa.com/topsites}}, from now on referenced as \textit{top domains snapshot}.
It contains the most popular one million web sites, regardless whether they are \texttt{.com} domains or not.
It was also retrieved on 2017/05/01.

In an effort to help with the reproducibility and replicability of this project, the original data and processing code used in this project has been made available in a public repository\footnote{\url{https://github.com/jsidrach/idn-homograph-attack}}.

\subsection{Data Preprocessing}
We used the \textit{.com snapshot} to obtain all the international domains of the \texttt{.com} domain zone.
All international domain names are represented using Punycode, so they start with the \texttt{xn--} prefix.
Using this information we first filtered the \textit{.com snapshot} to match only the \texttt{.com} domains that start with \texttt{xn--}.
Since the \textit{.com snapshot} contains name server records, a single domain may have more than one entry in the file.
Only the domain name is relevant for this project, so the rest of the columns of the \textit{.com snapshot} were discarded.
We sorted the domains in lexicographic order, removing duplicates in the process.

We used the \textit{top domains snapshot} as a source for the canonical (non-international) domain name of a website, based on its ranking.
As we were only interested in non-international domain names, so we discarded the ones that start with the preffix \texttt{xn--}.
We also removed sub-domains, as we only had top level domain names in the \textit{.com snapshot}.

\subsection{Clustering}
The underlying assumptions behind the clustering process are that homograph attacks are more likely to target popular domains, and that the million domains from the \textit{top domains snapshot} contains most of the popular domains.
To detect potential international domain name homograph attacks, we cluster the international domain names in the \textit{.com snapshot}.
The representative of each cluster is a homograph domain name from the \textit{top domains snapshot}.
Only clusters with more than one domain name are output.

The detection of homograph domain names is done using an algorithm to check if two strings are confusable.
This algorithm is described in the \textit{Unicode Technical Standard \#39}\footnote{\url{http://www.unicode.org/reports/tr39/#Confusable_Detection}}.
The approach followed has some caveats, that could be addressed in future work.
For instance, it does not detect a homograph of \texttt{www.google.com}, where the first ``.'' has been replaced by a similar looking unicode character.
More sophisticated homographs can also be generated by using unicode characters similar to ``/''.
It is also worth noting that only domain names from the \texttt{.com} zone were considered, but similar studies could be done to other top level domains following the same procedure.
All things considered, we still think this approach is a valuable first step.

Another type of clustering was also performed.
We grouped the homograph domain names by its registrant organization, and ranked each organization by the number of homograph domain names to their name.
This could shed light on which registrant organizations are allowing homograph domain names, or even on which indivuduals and companies are doing it the most.

\subsection{Manual Classification}
The last part of the data processing was the classification of the homograph domain names.
This classification was done manually, to be able to differentiate between scam and unrelated websites.
Some domains expired at the time of the classification, and were subsequently deleted from the output file.

The two high level categories that were defined are \textit{Canonical} and \textit{Third Party}.
The first one, \textit{Canonical}, is employed when the domain is registered by the same organization as its canonical homograph domain name.
The second one, \textit{Third Party}, is used when the domain is registered by a different organization than its canonical homograph domain name.
Additionally, a more detailed classification was made:
\itemize{
\setlength{\itemsep}{0pt}
\setlength{\parskip}{0pt}
\setlength{\parsep}{0pt}
\item \textit{Canonical - Parking}: domain is registered but not accessible via HTTP.
\item \textit{Canonical - Redirect}: domain redirects (HTTP Status Code 301/307/308) to its canonical homograph domain name.
\item \textit{Third Party - Redirect to Canonical:} domain redirects (HTTP Status Code 301/307/308) to its canonical homograph domain name.
\item \textit{Third Party - Unrelated}: domain resolves, but the contents of the website are totally unrelated to its canonical homograph domain name.
\item \textit{Third Party - Parking}: domain is registered, but no accessible via HTTP, or when accessed, a default domain parking webpage is displayed.
\item \textit{Third Party - Scam}: domain resolves, and the website displayed is a clear attempt (similar color, logos, etc.) to make users think they are visiting the canonical homograph domain name.
}

\section{Results}
TODO
- Comment different results obtained
- Explain consequences of caveats (low number of matches)
- Explain and reference tables
- Some other interesting results (topic-related domain hoarding)

\begin{table*}[]
\centering
\begin{tabular}{lrr}
\hline
Domains                                       & \#                         & \%                         \\ \hline
\itshape\sffamily{Canonical domain names}     & \itshape\sffamily{458731}  & \itshape\sffamily{8.31\%}  \\
\hspace{0.5cm} With IDN homographs            & 825                        & 6.04\%                     \\
\hspace{0.5cm} Without IDN homograpphs        & 457906                     & 2.27\%                     \\
\itshape\sffamily{International Domain Names} & \itshape\sffamily{1045400} & \itshape\sffamily{91.69\%} \\
\hspace{0.5cm} With canonical homograph       & 1099                       & 3.68\%                     \\
\hspace{0.5cm} Without canonical homograph    & 1044301                    & 2.74\%                     \\ \hline
\end{tabular}
\caption{Overview of the clustering results.}
\label{clustering-results}
\end{table*}

\begin{table}[]
\centering
\begin{tabular}{lr}
\hline
Domain       & \# of IDN homographs \\ \hline
google.com   & 24                   \\
youtube.com  & 3                    \\
facebook.com & 9                    \\
baidu.com    & 3                    \\
yahoo.com    & 4                    \\
reddit.com   & 1                    \\
qq.com       & 2                    \\
taobao.com   & 1                    \\
live.com     & 1                    \\
vk.com       & 6                    \\ \hline
\end{tabular}
\caption{Top ten .com domains in the Alexa ranking with IDN homographs.}
\label{top-com-alexa-with-idns}
\end{table}

\begin{table}[]
\centering
\begin{tabular}{lrr}
\hline
Status                               & \#                     & \%                         \\ \hline
\itshape\sffamily{Canonical}         & \itshape\sffamily{88}  & \itshape\sffamily{8.31\%}  \\
\hspace{0.5cm} Parking               & 64                     & 6.04\%                     \\
\hspace{0.5cm} Redirect              & 24                     & 2.27\%                     \\
\itshape\sffamily{Third Party}       & \itshape\sffamily{971} & \itshape\sffamily{91.69\%} \\
\hspace{0.5cm} Redirect to Canonical & 39                     & 3.68\%                     \\
\hspace{0.5cm} Unrelated             & 29                     & 2.74\%                     \\
\hspace{0.5cm} Parking               & 872                    & 82.34\%                    \\
\hspace{0.5cm} Scam                  & 31                     & 2.93\%                     \\ \hline
\end{tabular}
\caption{Breakdown of the manually classified homograph IDNs.}
\label{manually-classified-idns}
\end{table}

TODO: mention:
  Number of Third Party domains whose Registrant Organization and Email has than one homograph IDN registered: 437
  Number of Registrant's Organization and Email that have more than one homograph IDN: 82

\begin{table*}[]
\centering
\begin{tabular}{llr}
\hline
Registrant organization              & Registrant email           & \# of homograph IDNs \\ \hline
Domains By Proxy, LLC                & --                         & 89                   \\
Super Privacy Service c/o Dynadot    & privacy@dynadot.com        & 23                   \\
Domain Registries Foundation         & --                         & 22                   \\
Duong Thien                          & thiendv@outlook.com        & 18                   \\
Syngenuity Limited                   & manager@syngenuity.com     & 12                   \\
Helpnet: Brand Development \& Sales  & help@strongestbrands.com   & 12                   \\
ONUNO L.L.C.                         & corucas@gmail.com          & 11                   \\
Privacy Protection Service INC d/b/a & contact@privacyprotect.org & 10                   \\
Hubertus Henz                        & hu\_h5@yahoo.de            & 9                    \\
wuyu                                 & wy65535@126.com            & 7                    \\ \hline
\end{tabular}
\caption{Top ten registrants with the most homograph IDNs.}
\label{top-registrants-homograph-idns}
\end{table*}

\section{Ethical Considerations}
TODO
- Brief explanation why this research is ethical

\section{Conclusions}
TODO
- Conclusions of our work
- Possible future work
- TLDs

TODO DELETE~\cite{ipv4sta}

\section*{Acknowledgements}
We would like to thank Louis DeKoven and Stefan Savage for their help and support throughout this project.

{\footnotesize \bibliographystyle{acm}
\bibliography{bibliography}}

\end{document}
